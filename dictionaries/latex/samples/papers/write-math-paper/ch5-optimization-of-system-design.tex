%!TEX root = write-math-ba-paper.tex

\section{Optimization of System Design}\label{ch:Optimization-of-System-Design}
In order to evaluate the effect of different preprocessing algorithms, features
and adjustments in the \gls{MLP} training and topology, the following baseline
system was used:

Scale the recording to fit into a unit square while keeping the aspect ratio,
shift it as described in \cref{sec:preprocessing},
resample it with linear interpolation to get 20~points per stroke, spaced
evenly in time. Take the first 4~strokes with 20~points per stroke and
2~coordinates per point as features, resulting in 160~features which is equal
to the number of input neurons. If a recording has less than 4~strokes, the
remaining features were filled with zeroes.

All experiments were evaluated with four baseline systems $B_{hl=i}$, $i \in \Set{1,
2, 3, 4}$, where $i$ is the number of hidden layers as different topologies
could have a severe influence on the effect of new features or preprocessing
steps. Each hidden layer in all evaluated systems has $500$ neurons.

Each \gls{MLP} was trained with a learning rate of $\eta = 0.1$ and a momentum
of $\alpha = 0.1$. The activation function of every neuron in a hidden layer is
the sigmoid function. The neurons in the
output layer use the softmax function. For every experiment, exactly one part
of the baseline systems was changed.


\subsection{Random Weight Initialization}
The neural networks in all experiments got initialized with a small random
weight

\[w_{i,j} \sim U(-4 \cdot \sqrt{\frac{6}{n_l + n_{l+1}}}, 4 \cdot \sqrt{\frac{6}{n_l + n_{l+1}}})\]

where $w_{i,j}$ is the weight between the neurons $i$ and $j$, $l$ is the layer
of neuron $i$, and $n_i$ is the number of neurons in layer $i$. This random
initialization was suggested on
\cite{deeplearningweights} and is done to break symmetry.

This can lead to different error rates for the same systems just because the
initialization was different.

In order to get an impression of the magnitude of the influence on the different
topologies and error rates the baseline models were trained 5 times with
random initializations.
\Cref{table:baseline-systems-random-initializations-summary}
shows a summary of the results. The more hidden layers are used, the more do
the results vary between different random weight initializations.

\begin{table}[h]
    \centering
    \begin{tabular}{crrr|rrr} %chktex 44
    \toprule
    \multirow{3}{*}{System}  & \multicolumn{6}{c}{Classification error}\\
    \cmidrule(l){2-7}
               & \multicolumn{3}{c}{Top-1}   & \multicolumn{3}{c}{Top-3}\\
               & Min                   & Max                   & Mean                  & Min                  & Max                  & Mean\\\midrule
    $B_{hl=1}$ & $\SI{23.1}{\percent}$ & $\SI{23.4}{\percent}$ & $\SI{23.2}{\percent}$ & $\SI{6.7}{\percent}$ & $\SI{6.8}{\percent}$ & $\SI{6.7}{\percent}$ \\
    $B_{hl=2}$ & \underline{$\SI{21.4}{\percent}$} & \underline{$\SI{21.8}{\percent}$}& \underline{$\SI{21.6}{\percent}$} & $\SI{5.7}{\percent}$ & \underline{$\SI{5.8}{\percent}$} & \underline{$\SI{5.7}{\percent}$}\\
    $B_{hl=3}$ & $\SI{21.5}{\percent}$ & $\SI{22.3}{\percent}$ & $\SI{21.9}{\percent}$ & \underline{$\SI{5.5}{\percent}$} & $\SI{5.8}{\percent}$ & \underline{$\SI{5.7}{\percent}$}\\
    $B_{hl=4}$ & $\SI{23.2}{\percent}$ & $\SI{24.8}{\percent}$ & $\SI{23.9}{\percent}$ & $\SI{6.0}{\percent}$ & $\SI{6.4}{\percent}$ & $\SI{6.2}{\percent}$\\
    \bottomrule
    \end{tabular}
    \caption{The systems $B_{hl=1}$ -- $B_{hl=4}$ were randomly initialized,
             trained and evaluated 5~times to estimate the influence of random
             weight initialization.}
\label{table:baseline-systems-random-initializations-summary}
\end{table}


\subsection{Stroke connection}
In order to solve the problem of interrupted strokes, pairs of strokes
can be connected with stroke connection algorithm. The idea is that for
a pair of consecutively drawn strokes $s_{i}, s_{i+1}$ the last point $s_i$ is
close to the first point of $s_{i+1}$ if a stroke was accidentally split
into two strokes.

$\SI{59}{\percent}$ of all stroke pair distances in the collected data are
between $\SI{30}{\pixel}$ and $\SI{150}{\pixel}$. Hence the stroke connection
algorithm was evaluated with $\SI{5}{\pixel}$, $\SI{10}{\pixel}$ and
$\SI{20}{\pixel}$.
All models top-3 error improved with a threshold of $\theta = \SI{10}{\pixel}$
by at least $\num{0.2}$ percentage points, except $B_{hl=4}$ which did not notably
improve.


\subsection{Douglas-Peucker Smoothing}
The Douglas-Peucker algorithm was applied with a threshold of $\varepsilon =
0.05$, $\varepsilon = 0.1$ and $\varepsilon = 0.2$ after scaling and shifting,
but before resampling. The interpolation in the resampling step was done
linearly and with cubic splines in two experiments. The recording was scaled
and shifted again after the interpolation because the bounding box might have
changed.

The result of the application of the Douglas-Peucker smoothing with $\varepsilon
> 0.05$ was a high rise of the top-1 and top-3 error for all models $B_{hl=i}$.
This means that the simplification process removes some relevant information and
does not---as it was expected---remove only noise. For $\varepsilon = 0.05$
with linear interpolation some models top-1 error improved, but the
changes were small. It could be an effect of random weight initialization.
However, cubic spline interpolation made all systems perform more than
$\num{1.7}$ percentage points worse for top-1 and top-3 error.

The lower the value of $\varepsilon$, the less does the recording change after
this preprocessing step. As it was applied after scaling the recording such that
the biggest dimension of the recording (width or height) is $1$, a value of
$\varepsilon = 0.05$ means that a point has to move at least $\SI{5}{\percent}$
of the biggest dimension.


\subsection{Global Features}
Single global features were added one at a time to the baseline systems. Those
features were re-curvature
$\text{re-curvature}(stroke) = \frac{\text{height}(stroke)}{\text{length}(stroke)}$
as described in \cite{Huang06}, the ink feature which is the summed length
of all strokes, the stroke count, the aspect ratio and the stroke center points
for the first four strokes. The stroke center point feature improved the system
$B_{hl=1}$ by $\num{0.3}$~percentage points for the top-3 error and system $B_{hl=3}$ for
the top-1 error by $\num{0.7}$~percentage points, but all other systems and
error measures either got worse or did not improve much.

The other global features did improve the systems $B_{hl=1}$ -- $B_{hl=3}$, but not
$B_{hl=4}$. The highest improvement was achieved with the re-curvature feature. It
improved the systems $B_{hl=1}$ -- $B_{hl=4}$ by more than $\num{0.6}$~percentage points
top-1 error.


\subsection{Data Multiplication}
Data multiplication can be used to make the model invariant to transformations.
However, this idea seems not to work well in the domain of on-line handwritten
mathematical symbols. We tripled the data by adding a version that is rotated
3~degrees to the left and another one that is rotated 3~degrees to the right
around the center of mass. This data multiplication made all classifiers for
most error measures perform worse by more than $\num{2}$~percentage points for
the top-1 error.

The same experiment was executed by rotating by 6~degrees and in another
experiment by 9~degrees, but those performed even worse.

Also multiplying the data by a factor of 5 by adding two 3-degree rotated
variants and two 6-degree rotated variant made the classifier perform worse
by more than $\num{2}$~percentage points.


\subsection{Pretraining}\label{subsec:pretraining-evaluation}
Pretraining is a technique used to improve the training of \glspl{MLP} with
multiple hidden layers.

\Cref{table:pretraining-slp} shows that \gls{SLP} improves the classification
performance by $\num{1.6}$ percentage points for the top-1 error and
$\num{1.0}$ percentage points for the top-3 error. As one can see in
\cref{fig:training-and-test-error-for-different-topologies-pretraining}, this
is not only the case because of the longer training as the test error is
relatively stable after $\num{1000}$ epochs of training. This was confirmed
by an experiment where the baseline systems where trained for $\num{10000}$
epochs and did not perform notably different.

\begin{figure}[htb]
    \centering
    \begin{tikzpicture}
    \begin{axis}[
            axis x line=middle,
            axis y line=middle,
            enlarge y limits=true,
            xmin=0,
            % xmax=1000,
            ymin=0.18, ymax=0.4,
            minor ytick={0, 0.01, ..., 1},
            % width=15cm, height=8cm,     % size of the image
            grid = both,
            minor grid style={dashed, gray!30},
            major grid style={gray!40},,
            %grid style={dashed, gray!30},
            ylabel=error,
            xlabel=epoch,
            legend cell align=left,
            legend style={
                at={(0.5,-0.1)},
                anchor=north,
                legend columns=2
            }
         ]
          \addplot[mark=x,green] table [each nth point=20,x=epoch, y=testerror, col sep=comma] {baseline-1.csv};
          \addplot[mark=x,orange] table [each nth point=20,x=epoch, y=testerror, col sep=comma] {baseline-2.csv};
          \addplot[mark=x,red] table [each nth point=20,x=epoch, y=testerror, col sep=comma] {baseline-2-pretraining.csv};
          \legend{{1 hidden layer},
                  {2 hidden layers},
                  {2 hidden layers with pretraining}}
    \end{axis}
\end{tikzpicture}
    \caption{Training- and test error by number of trained epochs for different
             topologies with \acrfull{SLP}. The plot shows
             that all pretrained systems performed much better than the systems
             without pretraining. All plotted systems did not improve
             with more epochs of training.}
\label{fig:training-and-test-error-for-different-topologies-pretraining}
\end{figure}

\begin{table}[tb]
    \centering
    \begin{tabular}{lrrrr}
    \toprule
    \multirow{2}{*}{System}  & \multicolumn{4}{c}{Classification error}\\
    \cmidrule(l){2-5}
                & Top-1                  & Change               & Top-3                & Change                 \\\midrule
    $B_{hl=1}$     & $\SI{23.2}{\percent}$  & -                    & $\SI{6.7}{\percent}$ & - \\
    $B_{hl=2,SLP}$ & $\SI{19.9}{\percent}$ & $\SI{-1.7}{\percent}$ & $\SI{4.7}{\percent}$ & $\SI{-1.0}{\percent}$\\
    $B_{hl=3,SLP}$ & \underline{$\SI{19.4}{\percent}$} & $\SI{-2.5}{\percent}$ & \underline{$\SI{4.6}{\percent}$} & $\SI{-1.1}{\percent}$\\
    $B_{hl=4,SLP}$ & $\SI{19.6}{\percent}$ & $\SI{-4.3}{\percent}$ & \underline{$\SI{4.6}{\percent}$} & $\SI{-1.6}{\percent}$\\
    \bottomrule
    \end{tabular}
    \caption{Systems with 1--4 hidden layers which used \acrfull{SLP}
             compared to the mean of systems $B_{hl=1}$--$B_{hl=4}$ displayed
             in \cref{table:baseline-systems-random-initializations-summary}
             which used pure gradient descent. The \gls{SLP}
             systems clearly performed worse.}
\label{table:pretraining-slp}
\end{table}


Pretraining with denoising auto-encoder lead to the much worse results listed in
\cref{table:pretraining-denoising-auto-encoder}. The first layer used a $\tanh$
activation function. Every layer was trained for $1000$ epochs and the
\gls{MSE} loss function. A learning-rate of $\eta = 0.001$, a corruption of
$\varkappa = 0.3$ and a $L_2$ regularization of $\lambda = 10^{-4}$ were
chosen. This pretraining setup made all systems with all error measures perform
much worse.

\begin{table}[tb]
    \centering
    \begin{tabular}{lrrrr}
    \toprule
    \multirow{2}{*}{System}  & \multicolumn{4}{c}{Classification error}\\
    \cmidrule(l){2-5}
                 & Top-1                  & Change               & Top-3                & Change                 \\\midrule
    $B_{hl=1,AEP}$ & $\SI{23.8}{\percent}$ & $\SI{+0.6}{\percent}$ & $\SI{7.2}{\percent}$ & $\SI{+0.5}{\percent}$\\
    $B_{hl=2,AEP}$ & \underline{$\SI{22.8}{\percent}$} & $\SI{+1.2}{\percent}$ & $\SI{6.4}{\percent}$ & $\SI{+0.7}{\percent}$\\
    $B_{hl=3,AEP}$ & $\SI{23.1}{\percent}$ & $\SI{+1.2}{\percent}$ & \underline{$\SI{6.1}{\percent}$} & $\SI{+0.4}{\percent}$\\
    $B_{hl=4,AEP}$ & $\SI{25.6}{\percent}$ & $\SI{+1.7}{\percent}$ & $\SI{7.0}{\percent}$ & $\SI{+0.8}{\percent}$\\
    \bottomrule
    \end{tabular}
    \caption{Systems with denoising \acrfull{AEP} compared to pure
             gradient descent. The \gls{AEP} systems performed worse.}
\label{table:pretraining-denoising-auto-encoder}
\end{table}
